%----------------------------------------------------------------------------
%
% Destruction
% 
%----------------------------------------------------------------------------
%----------------------------------------------------------------------------
%----------------------------------------------------------------------------

\ProvidesFile{destruction.tex}
\documentclass[review]{acmsiggraph-job}
\usepackage{amsmath}
\usepackage{amssymb}
\usepackage{wasysym}
\usepackage[scaled=.92]{helvet}
\usepackage{times}
\usepackage{graphicx}
\usepackage{parskip}
\usepackage{url}
\usepackage[labelfont=bf,textfont=it]{caption}
\usepackage{color}
\TOGonlineid{papers\_????}

%\TOGyear{2013}
%\TOGmonth{November}
%\TOGvenue{ACM SIGGRAPH Asia}
%\TOGlocation{Hong Kong}
%\TOGvolume{32}
%\TOGnumber{6}
%\TOGarticlenum{185}
%\TOGarticleDOI{2508363.2508430}

%\TOGprojectURL{http://sealab.cs.utah.edu/Papers/Gerszewski-2013-PAO/index.html}
%\TOGvideoURL{http://sealab.cs.utah.edu/Papers/Gerszewski-2013-PAO/Stuff/Gerszewski-2013-PAO.mov}

\setlength\paperwidth{8.5in}  % Override any random settings...
\setlength\paperheight{11in}

\title{Physics-based Animation of Destruction}

\author{}
\pdfauthor{}
\pdftitle{Physics-based Animation of Destruction}

\newcommand{\theKeywords}{}

\begin{document}
%\abovecopyrightspacetext{}

%\teaser{
%\includegraphics[width=\linewidth]{Figures/Blank}
%\caption{}
%\label{fig:Blank}
%}

\maketitle

\begin{abstract}
\end{abstract}
\begin{CRcatlist}
  \CRcat{I.3.7}{Computer Graphics}{Three-Dimensional Graphics and Realism}{Animation};
  \CRcat{I.6.8}{Simulation and Modeling}{Types of Simulation}{Animation}.
\end{CRcatlist}

\keywordlist{\theKeywords}

\TOGlinkslist

%----------------------------------------------------------------------------
%----------------------------------------------------------------------------

\definecolor{AdamColor}{rgb}{0,0,0.7}
\newcommand{\adam}[1]{{\color{AdamColor} #1}}
%\newcommand{\adam}[1]{{#1}}

%\let\shortcite=\cite
%\newcommand{\shortcite}[1]{\cite{#1}}
\newcommand{\etal}{and colleagues}
\newcommand{\Mueller}{M\"uller}
\newcommand{\BM}[1]{\B{#1}}
%\newcommand{\B}[1]{\mbox{\boldmath$#1$}}
%\newcommand{\B}[1]{\textbf{\textit{#1}}}
\newcommand{\B}[1]{\mathit{\mathbf{#1}}}
\newcommand{\Per}{\%}
\newcommand{\Unit}[1]{{\mbox{$\,\mathrm{#1}$}}}
\newcommand{\Snit}[1]{{\mbox{\small$\mathrm{#1}$}}}
\newcommand{\Tr}[1]{\mathrm{Tr}\left(#1\right)}
\newcommand{\Hz}{\Unit{Hz}}
\newcommand{\MHz}{\Unit{MHz}}
\newcommand{\GHz}{\Unit{GHz}}
\newcommand{\Sec}{\Unit{sec}}
\newcommand{\SPF}{\Unit{sec/frame}}
\newcommand{\Min}{\Unit{min}}
\newcommand{\Max}{\Unit{max}}
\newcommand{\M}{\Unit{m}}
\newcommand{\Nab}{\B{\nabla}}
\newcommand{\TP}{^\mathsf{T}}

\newcommand{\Dist}{\mbox{dist}}

\newcommand{\figureTopBot}[1]{
  \begin{figure}[!tb]{\sloppy #1}\end{figure}
}

\newcommand{\figureTop}[1]{
  \begin{figure}[!t]{\sloppy #1}\end{figure}
}
 
\newcommand{\figureBot}[1]{
  \begin{figure}[!b]{\sloppy #1}\end{figure}
}

\newcommand{\figureWideTop}[1]{
  \begin{figure*}[!t]{\sloppy #1}\end{figure*}
}

\newcommand{\eqAlgn}{\!\!&\!\!}

\newcommand{\Eref}[1]{Equation~(\ref{#1})}
\newcommand{\Erefs}[2]{Equations~(\ref{#1}) and (\ref{#2})}
\newcommand{\eref}[1]{Equation~(\ref{#1})}
\newcommand{\erefs}[2]{Equations~(\ref{#1}) and (\ref{#2})}
\newcommand{\Sref}[1]{Section~\ref{#1}}
\newcommand{\sref}[1]{Section~\ref{#1}}
\newcommand{\fref}[1]{Figure~\ref{#1}}
\newcommand{\frefAND}[2]{Figures~\ref{#1} and~\ref{#2}}
\newcommand{\frefs}[2]{Figures~\ref{#1} and~\ref{#2}}
\newcommand{\frefss}[3]{Figures~\ref{#1}, \ref{#2}, and~\ref{#3}}
\newcommand{\frefsss}[4]{Figures~\ref{#1}, \ref{#2}, \ref{#3}, and~\ref{#4}}
\newcommand{\Fref}[1]{Figure~\ref{#1}}
\newcommand{\Frefs}[2]{Figures~\ref{#1} and~\ref{#2}}
\newcommand{\Frefss}[3]{Figures~\ref{#1}, \ref{#2}, and~\ref{#3}}
\newcommand{\Frefsss}[4]{Figures~\ref{#1}, \ref{#2}, \ref{#3}, and~\ref{#4}}
\newcommand{\tref}[1]{Table~\ref{#1}}

\renewcommand{\labelenumi}{\arabic{enumi}.}
\renewcommand{\labelenumii}{\alph{enumii}.}
\renewcommand{\labelenumiii}{\roman{enumiii}.}

\newenvironment{algstep}{%
  \begin{enumerate}%
    \setlength{\itemsep}{0in}%
    \setlength{\partopsep}{0in}%
    \setlength{\topsep}{0in}%
}{\end{enumerate}}

%----------------------------------------------------------------------------
%----------------------------------------------------------------------------

\section{Introduction}\label{sec:Introduction}
\copyrightspace

\section{Related Work}\label{sec:RelatedWork}

\section{Methods}
We envision three-step simulation:
\begin{enumerate}
\item update momentum
\item handle collisions/contact
\item update positions
\end{enumerate}

As a point of comparison, Tamar did 
\begin{enumerate}
\item update momentum
\item handle collisions/contact
\item update positions
\item update momentum again (this time incorporating constraints in the solve).
\end{enumerate}
The two updates are for their Newmark integration, but I think that is overkill.  One update should be enough.
I am unclear on how they are incorporating the constraints in the second solve.  They claim this is a
projection and that care must be used, but I cannot see how this would actually be done.  Also,
it is not clear, but I suspect that they use the fem node masses when projecting the constraints, which
I think would give much greater weight to the rigid bodies.

\subsection{Update Momentum}
Currently this is done separately for the finite elements and the rigid bodies.  But, Tamar's paper
indicates that some consideration of the constraints during the (second) momentum update is necessary
and I think she is probably right unless very small timesteps are used.  We were originally thinking 
of applying the constraints after collisions, but I think this coupling would be too loose.  It would also
suffer from the {\em stretched spring} problem where the constrained fem nodes are much more stretched than
their neighbors.  One thought is to solve a coupled system.  

Let $\B{M}_{d}$ and $\B{K}_{d}$ be the fem mass and stiffness matrices and assume that the rigid body 
mass and intertia tensors are stacked in $\B{M}_r$.  Also, let $\B{C}$ represent the constraints.  Then basically
we would solve 
\begin{equation}
\left(\begin{array}{cc}
\left(
\begin{array}{cc}
\B{M}_d+\B{K}_d & 0 \\
0 & \B{M}_r 
\end{array}
\right)
& \B{C}^T\\
\B{C} & 0
\end{array}
\right)\left(
\begin{array}{c}
\B{v}_d^*\\
\B{v}_r^*\\
\B{\lambda}
\end{array}
\right) =
\left(
\begin{array}{c}
\B{Mv}_d + f\\
\B{Mv}_r\\
0
\end{array}
\right)
\end{equation}
There seem to be several advantages to this approach. (1) It explicitly couples the rigid and deformable bodies, allowing
information on the constraints to cascade throught the finite element mesh (avoiding the stretchy springs artifacts).  
(2) Nobody has done it this way before.  Disadvantages: the system would be huge!  And it is indefinite (we could make
it definite by changing the lower right 0 to an identity, encouraging the system to minimize the lagrange multipliers).  
It also would not quite enforce the constraints.

Other options would be to figure out the project that Tamar was using (but this might lead to the stretchy springs).  
Another option would be to treat the rigid bodies velocities as constraints on the finite element simulation (analogous to time-splitting schemes).
We could then use the Lagrange multipliers to compute equal-and-oposite forces on the rigid bodies.  This feels less elegant to 
me, but allows a cleaner separation between the rigid body and deformable codes.  Not sure exactly what the timestep would look like to make
sure that constraints are enforced at render-time.

Another approach would be to simply constrain fem vertices attached to rigid bodies to have rigid motion, either by using reduced coordinates
or Lagrange multipliers.  This would ignore the rigid bodies mass/momentum though.

\subsection{Collision Detection}
Tamar used Guendelman 2002 (iterative impulse-based collision handling based on candidate positions) for rigid-rigid and deformable-rigid.
Deformables were treated as particles.  Deformable-deformable were treated later in the timestep using Bridson's cloth code.
As I said in my email, I don't advocate using just the surface of a volumetric mesh---it doesn't work very well for reasonable time steps or
coarse models.  One thing we could try is to treat the deformable objects as rigid, using some ``shape'' (i.e. throwing out rotation/translation)
over the timestep (maybe the end?) and integrating the linear and angular momentum.  Then just throw everything into bullet.  I don't really have
any other ideas.  I am not familiar enough with bullet to know if this would be efficient.

\subsection{Update Positions}
This is straightforward.  We might need some sort of post-stabilization.

\section{Results and Discussion}\label{sec:Results}

\paragraph{Limitations and Future Work}

\section*{Acknowledgments}


\bibliographystyle{acmsiggraph}
\bibliography{uniFluid}

\end{document}
